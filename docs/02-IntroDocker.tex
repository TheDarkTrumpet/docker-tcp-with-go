\documentclass{beamer}
\usepackage{minted}
\setbeamertemplate{navigation symbols}{}


\usetheme{Montpellier}

\beamersetuncovermixins{\opaqueness<1>{25}}{\opaqueness<2->{15}}
\begin{document}
\title{Introduction to Docker}
\author{David Thole}
\date{\today} 
\begin{frame}
  \titlepage
\end{frame}

\begin{frame}{Outline}
  \tableofcontents
\end{frame} 


\section{Docker Overview}
\subsection{Docker Definition}
\begin{frame}\frametitle{Docker Definition}
  Docker is defined, by Microsoft \small{[1]} as:
  \begin{quote}
    Docker is an open-source project for automating the deployment of applications as portable, self-sufficient containers that can run on the cloud or on-premises.
  \end{quote}
\end{frame}
\subsection{Why Use Docker}
\begin{frame}\frametitle{Why use Docker?}
  There are three main ways that Docker can benefit you, both at home, and in the enterprise:
  \begin{itemize}
  \item \textbf{Testing} - The ability to test software
  \item \textbf{Consisting Deployment} - Same ``image'' used in QA, Prod.
  \item \textbf{Simplified Development} - Less local dependencies, easier to spin up infrastructure.
  \end{itemize}
\end{frame}
\subsection{Architecture}
\begin{frame}\frametitle{Docker Architecture}
  \begin{figure}
    \includegraphics[width = 1\textwidth]{02-DockerVsVMs.png}
  \end{figure}
\end{frame}
\begin{frame}\frametitle{Docker Architecture}
  \begin{figure}
    \includegraphics[width = 1\textwidth]{02-DockerRunArch.png}
  \end{figure}
\end{frame}
\begin{frame}\frametitle{Docker Architecture}
  \begin{figure}
    \includegraphics[width = 1\textwidth]{02-DockerNetwork.png}
  \end{figure}
\end{frame}

\section{Pulling and Running Containers}
\subsection{Docker Pull}
\begin{frame}\frametitle{Docker Pull}
  \textbf{Purpose:} To pull an image from a repository.  Default is hub.dockerhub.com
  \\ \\
  \mintinline{shell}{docker pull [image]}
\end{frame}
\subsection{Docker Run}
\begin{frame}\frametitle{Docker Run}
  \textbf{Purpose:} Takes a currently existing image, or pulls a new image with name, and spawns a new container.  Many options exist, such as exposing ports, linking to other containers, etc.
  \\ \\
  \mintinline{shell}{docker run [opts] [image]}
\end{frame}
\subsection{Docker Kill}
\begin{frame}\frametitle{Docker Kill}
  \textbf{Purpose:} Will stop execution of a currently running container.
  \\ \\
  \mintinline{shell}{docker kill [hash or name]}  
\end{frame}
\subsection{Docker Rm, and RMI}
\begin{frame}\frametitle{Docker Rm, and Rmi}
  \textbf{Purpose:} Will remove a container (rm) or image (rmi)
  \\ \\
  \mintinline{shell}{docker rm [container hash or name]}  \\
  \mintinline{shell}{docker rmi [image name]}  \\  
\end{frame}

\section{References}
\begin{frame}\frametitle{References}
  \begin{itemize}
    \item [1] - https://docs.microsoft.com/en-us/dotnet/architecture/microservices/container-docker-introduction/docker-defined
  \end{itemize}
\end{frame}

\end{document}
