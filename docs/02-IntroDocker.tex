% This text is proprietary.
% It's a part of presentation made by myself.
% It may not used commercial.
% The noncommercial use such as private and study is free
% Dec 2007
% Author: Sascha Frank 
% University Freiburg 
% www.informatik.uni-freiburg.de/~frank/
%
% 
\documentclass{beamer}
\setbeamertemplate{navigation symbols}{}


\usetheme{Montpellier}

\beamersetuncovermixins{\opaqueness<1>{25}}{\opaqueness<2->{15}}
\begin{document}
\title{Introduction to Docker}
\author{David Thole}
\date{\today} 
\begin{frame}
\titlepage
\end{frame}

\begin{frame}\frametitle{Table of contents}\tableofcontents
\end{frame} 


\section{Docker Overview}
\begin{frame}\frametitle{Docker Definition}
  Docker is defined, by Microsoft \small{[1]} as:
  \begin{quote}
    Docker is an open-source project for automating the deployment of applications as portable, self-sufficient containers that can run on the cloud or on-premises.
  \end{quote}
\end{frame}
\begin{frame}\frametitle{Docker Architecture}
  \begin{figure}
    \includegraphics[width = 1\textwidth]{02-DockerVsVMs.png}
  \end{figure}
\end{frame}
\begin{frame}\frametitle{Docker Architecture}
  \begin{figure}
    \includegraphics[width = 1\textwidth]{02-DockerRunArch.png}
  \end{figure}
\end{frame}
\begin{frame}\frametitle{Docker Architecture}
  \begin{figure}
    \includegraphics[width = 1\textwidth]{02-DockerNetwork.png}
  \end{figure}
\end{frame}

\section{Pulling and Running Containers}
\begin{frame}\frametitle{Docker Commands}

\end{frame}

\begin{frame}\frametitle{Docker Pull}
\end{frame}

\begin{frame}\frametitle{Docker Run}
\end{frame}

\begin{frame}\frametitle{Docker Kill}
\end{frame}

\begin{frame}\frametitle{Docker Rm, and Rmi}
\end{frame}

\section{References}
\begin{frame}\frametitle{References}
  \begin{itemize}
    \item [1] - https://docs.microsoft.com/en-us/dotnet/architecture/microservices/container-docker-introduction/docker-defined
  \end{itemize}
\end{frame}

\end{document}
