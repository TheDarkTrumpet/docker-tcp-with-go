\documentclass[legal,12pt,oneside,pdflatex,final,twocolumn]{article}
\usepackage[utf8]{inputenc}
\usepackage{booktabs}
\usepackage{fancyhdr}
\usepackage{hyperref}
\usepackage{parallel}
\usepackage{minted}
\usemintedstyle{vs}
\usepackage[export]{adjustbox}
\usepackage[margin=0.5in]{geometry}
\addtolength{\topmargin}{0in}

\usepackage{libertine}
\renewcommand*\familydefault{\sfdefault}  %% Only if the base font of the document is to be sans serif
\usepackage[T1]{fontenc}

\title{Using Docker (at UFG)}
\author{David Thole}
\date{Jan 05, 2022}

\begin{document}

\pagestyle{fancy}

\lhead{Lunch and Learn}
\chead {Jan 05, 2022}
\rhead{Using Docker}

\onecolumn

\begin{figure}
\begin{minipage}{0.47\textwidth}
\section{Resources}
    \begin{itemize}
        \item \textbf{Github Repo}: \\ \url{https://bit.ly/3qJtZOX}
        \item \textbf{DockerHub Repo}: \\ \url{https://dockr.ly/344M828}
        \item \textbf{Command Cheatsheet}: \\ \url{https://bit.ly/3pK6lme}
        \item \textbf{DockerHub}: \\ \url{https://hub.docker.com}
        \item \textbf{Docker Install Instructions}: \\ \url{https://dockr.ly/3EW5of5}
    \end{itemize}
\end{minipage}
\hfill
\begin{minipage}{0.47\textwidth}
  \centering
  \includegraphics[width=0.5\textwidth,right]{XX-DockerLogo.png}
\end{minipage}
\end{figure}

\section{Terms}
\begin{itemize}
  \item \textbf{Registry}: The storage location for images (think like Github), most popular is DockerHub, but private registries can be made
  \item \textbf{Image}: An immutable version of a ``stack'', defined as the software stack required for the particular file.
  \item \textbf{Container}: The currently executing image, that has state added.
  \item \textbf{Dockerfile}: A file that describes how an image should be created.
  \item \textbf{Docker Compose File}: A .yml file that describes how a collection of images should be invoked.
\end{itemize}

\section{Common Commands}
\subsection{Executing Containers}
\begin{itemize}
  \item \mintinline{shell}{docker run [container_name]} : Runs a container, non-daemon.
  \item \mintinline{shell}{docker run -d [container_name]} : Runs a container, daemon.
  \item \mintinline{shell}{docker-compose up <-d>} : Runs a compose collection, optional daemon.
\end{itemize}
\subsection{Investigating Containers}
\begin{itemize}
  \item \mintinline{shell}{docker ps} : Lists running containers.
  \item \mintinline{shell}{docker logs [container_name_or_hash]} : Runs a container, non-daemon.
\end{itemize}
\subsection{Killing/Cleanup of Containers}
\begin{itemize}
  \item \mintinline{shell}{docker kill [container_name_or_hash]} : Kills a container.
  \item \mintinline{shell}{docker stop [container_name_or_hash]} : Stops a container (graceful kill).
  \item \mintinline{shell}{docker-compose down} : Stops a compose collection.
  \item \mintinline{shell}{docker rm} : Removes a stopped container (kept around due to daemon).
  \item \mintinline{shell}{docker container prune} : Removes stopped containers.
\end{itemize}
\subsection{Working with Images}
\begin{itemize}
  \item \mintinline{shell}{docker images} : Lists available images.
  \item \mintinline{shell}{docker pull [image_name]} : Pulls an image from a repository.
  \item \mintinline{shell}{docker push [image_name]} : Pushes an image.
  \item \mintinline{shell}{docker rmi [image_name_or_hash]} : Removes an image locally.
  \item \mintinline{shell}{docker image prune} : Removes dangling images (images without a tag).
\end{itemize}
\end{document}
%%% Local Variables:
%%% mode: LaTeX
%%% TeX-master: t
%%% TeX-command-extra-options: "-shell-escape"
%%% End: